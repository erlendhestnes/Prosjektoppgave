%This is chapter 1
%%=========================================
\chapter{Introduction}

\section{Internet of Things (IoT)}

The IoT has a total potential economic impact of \$3.9 trillion to \$11.1 trillion a year by 2025. At the top end, that level of value—including the consumer surplus—would be equivalent to about 11 percent of the world economy \cite{mckinsey15}. Naturally, a vast amount of semiconductor companies want to be a part of this huge market, and a lot of these companies are therefore increasing their product portfolio to be ready to meet the IoT demand. 

\subsection{What Is IoT}

The Internet of Things (IoT) refers to the ever-growing network of physical objects that feature an IP address for Internet connectivity, and the communication that occurs between these objects and other Internet-enabled devices and systems \cite{webopedia}. Three very important keywords when it comes to IoT is connectivity, ease-of-use and low-power. A successful realization of the IoT movement is accelerated by creating solutions that are practical and easy to deploy ubiquitously. Low power, reliable wireless sensor networks translates into no wires and no worries for both customers and developers \cite{embedded_IoT}. 

\subsection{Problem With IoT}

It is quite reasonable to assume that you eventually will have more than a hundred IoT devices in your home. Say that each of these devices are using a battery, and that the average battery time for each device is about one or two years. Then you would have a huge job at hand for changing the batteries as they eventually run out of power. For the IoT market to be successful and fully appreciated by the customers, one need to be able make devices that does not require the user ever changing the battery. This is certainly a possibility as semiconductor companies continues to push the boundaries for ultra low power components.  

\subsection{Thesis outline}

This thesis focuses on a big part of IoT-applications, namely sensors. In it's broadest definition, a sensor is an object whose purpose is to detect events or changes in its environment, and then provide a corresponding output \cite{wikipedia_sensors}. They are used in vast number of applications and are considered to be a fundamental building block of the IoT. This project thesis targets a specific type of sensor that is used to detect motion, namely accelerometers. The thesis presents a thorough analysis of commercially available MEMS 3-axis accelerometers. The analysis focuses primarily on power consumption, but other considerations are also taken into account. Based on this analysis, an accelerometer is chosen to be used in a demo application.

The motivation behind this thesis is to show some of the potential IoT applications that can be achieved by using low-power features, found on modern microcontrollers, to acquire data from an ultra-low power accelerometer.

\begin{itemize}
\item Short description of each chapter:
\item Chapter 1: Some background theory...
\end{itemize}

\newpage

