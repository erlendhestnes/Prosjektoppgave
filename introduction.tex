%This is chapter 1
%%=========================================
\chapter{Introduction}

\section{Internet of Things (IoT)}

\textit{The IoT has a total potential economic impact of \$3.9 trillion to \$11.1 trillion a year by 2025. At the top end, that level of value - including the consumer surplus - would be equivalent to about 11 percent of the world economy} \cite{mckinsey15}.

%Naturally, a vast amount of semiconductor companies want to be a part of this huge emerging market, and a lot of these companies are therefore increasing their product portfolio to be ready to meet the IoT demand. 

%What this exactly means will be further

%However, several challenges need to be solved before the full potential of the IoT can be unlocked.

%\subsection{What is the IoT?}

The Internet of Things (IoT) concerns the continuous expansion of network enabled devices that feature an IP address for Internet connectivity, as well as the communication that exist between them and other Internet-enabled devices and systems \cite{webopedia}. Three important keywords when it comes to IoT is connectivity, ease-of-use and low-power. For a successful IoT realization, it is necessary to create solutions that are both practical and easy to deploy ubiquitously \cite{embedded_IoT}. %This can be achieved by creating both low power and reliable wireless sensor networks \cite{embedded_IoT}. 

%\subsection{A problem with IoT}

It is projected that there will be around 50 billion smart objects connected to the Internet of Things by the year of 2020 \cite{jayakumar14}. It is therefore reasonable to assume that there will be more than a hundred IoT devices in an average modern home. If each of these devices are using a battery, and if the average battery life for each device is about one or two years. One would need to change batteries several times per week. For the IoT market to be successful and fully appreciated by the customers, one need to be able make devices that does not require the user ever changing the battery. This is certainly a possibility as semiconductor companies continues to push the boundaries for ultra-low power components.  

%\subsection{Sensors for IoT}

This specialization project originates from a paper called \textit{Powering the Internet of Things}, written by \cite{jayakumar14}. That paper addresses some of the fundamental challenges of designing ultra-low power hardware platforms for the IoT. This thesis choose to look more closely at one of the common hardware components of an IoT connected device, namely sensors. A sensor is, in its broadest definition, a device that is both able to detect and respond to some type of input from the physical environment \cite{wigmore12}. They are used in large number of applications and are considered to be a fundamental building block of the IoT \cite{jayakumar14}. This thesis targets a specific type of sensor that is used to detect acceleration. 

%The thesis presents a thorough analysis of five commercially available microelectromechanical (MEMS) 3-axis accelerometers. The analysis focuses on finding the most low power device that is also suited for broad range of IoT applications. From this analysis, an accelerometer is chosen to be used for a custom reference board. Using modern embedded peripherals

%The thesis also focuses a great deal on creating a power efficient data acquisition scheme for this

%This board will be used in a later report to further explore different IoT applications that could benefit from having an ultra low power accelerometer.


\section{Goals and main contributions}

There is still a lot of challenges that need to be solved before the full potential of the IoT can be unleashed. It is therefore an excellent opportunity to confront and try to solve some of these problems, such that new applications can emerge and help accelerate the IoT trend. This thesis choose to further investigate one of the problems, namely the power consumption for motion enabled devices. The primary goal for this work has been to identify and explore the most low power accelerometers that are available on the market today, as well as to study some of the embedded data acquisition techniques that can be used to acquire data at the lowest possible power budget. A secondary objective has been to investigate possible IoT applications that can be realized with such ultra-low power accelerometers.
%fundamental principles and limitations behind these products. 

This thesis lay the theoretical groundwork for a possible later master thesis. The exploration of microelectromechanical (MEMS) accelerometers has given valuable insight into the physical limitations of the technology, with regards to mechanical noise, precision and power consumption. Different power saving peripherals inside a modern microcontroller has been investigated as well. The insight gained from this exploration has been used to design a reference board that employ a state of the art data acquisition scheme. Three IoT applications has been suggested for this reference board, and some of these applications may be the starting point of a future master thesis.

\newpage

\section{Thesis Outline}

The rest of this thesis is organized as follows:

\textbf{Chapter 2 - Background Theory:} This section presents relevant background theory regarding accelerometers, microelectromechanical systems (MEMS) and common low power techniques.  

\textbf{Chapter 3 - Accelerometer Overview:} A comparison of five ultra-low power, commercially available MEMS accelerometers

\textbf{Chapter 4 - Reference Board:} This chapter presents a reference board, as well as a data acquisition scheme, for developing motion enabled IoT applications.

\textbf{Chapter 5 - Accelerometer Analysis and Discussion:} Presents an analysis of the accelerometers presented in Chapter 3, and a discussion around choosing one of them for the proposed reference board in Chapter 4.

\textbf{Chapter 6 - Motion Based IoT Applications:} Discusses three potential IoT applications for the reference board presented in Chapter 4.

\textbf{Chapter 7 - Conclusion:} The thesis is concluded.

