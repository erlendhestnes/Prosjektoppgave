%This is chapter 1
%%=========================================
\chapter{Introduction}

The IoT has a total potential economic impact of \$3.9 trillion to \$11.1 trillion a year by 2025. At the top end, that level of value—including the consumer surplus—would be equivalent to about 11 percent of the world economy \cite{mckinsey15}.

Naturally, a vast amount of semiconductor companies want to be a part of this huge market.

\subsection{What Is IoT}

The Internet of Things (IoT) refers to the ever-growing network of physical objects that feature an IP address for Internet connectivity, and the communication that occurs between these objects and other Internet-enabled devices and systems \cite{webopedia}.

So let's say that you have more than a hundred IoT devices in your home, and that each of these devices are using a battery. Even if the average battery time for each device is one or two years, you would still have a huge job at hand for changing the batteries as they eventually run out of power. For the IoT market to be successful and fully appreciated by the customers, one need to be able make devices that does not require the user ever changing the battery. This is certainly a possibility as semiconductor companies continues to push the boundaries for ultra low power components.  

\newpage

\subsubsection{Energy Harvesting}

[This will be moved]

A battery will always run out of power, no matter how little current the system draws. So to be able to make a system that can continue indefinitely, one need to be able to have an infinite power source. Unfortunately, this does not exist. There is however a lot of research going into developing solutions that can turn energy from our surroundings into electrical energy. The principal is called energy harvesting, and it is truly a term that goes hand in hand with the IoT movement. Energy can be harvested from vibrations, sunlight, heat and a lot of other sources. A product that can implement one or more of these energy harvesting techniques has the potential to run indefinitely. Energy harvesting is something that has existed for a long time, but it is only with today's ultra low power components that it is actually possible to use this energy to power our devices.