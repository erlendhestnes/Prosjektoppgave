\chapter{Possible IoT Applications for Ultra Low Power Accelerometers}

This chapter presents possible IoT applications for low power, consumer grade accelerometers. The chapter is meant to illustrate the real world necessities for low power sensors and power efficient data acquisition techniques.

\section{Types of measurement}

Consumer grade MEMS accelerometers can typically measure from a range of $\pm2g$ to a range of $\pm16g$. This measurement range is perfectly suited for most use-cases. To put things in perspective, we can look at the acceleration of a few different events (from \cite{g-force}). Light vibrations for instance are in the order of 0.014 - 0.039g's, while the highest recorded earthquake was at its peak 2.99g's. The Bugatti Veyron supercar pulls 1.55g's when doing 0-100km/h in 2.4 seconds, while a F-16 fighter jet is at most able to pull 9g-12g for limited amount of time. A baseball struck by a baseball bat is on the other hand in the order of 3000g's, which is way above the measurement range for these devices. An acceleration of such magnitude may even in some cases destroy the MEMS element. 

The most relevant measurement range for IoT application are in the range of $\pm2g$ to a range of $\pm4g$. In this range we find both motion and vibration. The former is defined in this thesis as a slow moving event such as the movement of game controller or smart phone, while the latter is defined as oscillations that occur about an equilibrium point.

\subsection{Motion Detection}

Motion detection can in it simplest form be used as a trigger to wake an external system, this solution can for example be a viable substitute for on/off switches. For such a configuration we may only care about power consumption, since the collected data is not going to be used for any control input. For such an application it could be beneficial to use a special purpose device. For an application where control input is a requirement as well the motion activated trigger functionality, we would be interested in an accelerometer that uses low power in wake-up mode, but are able to sample with precision when it is required.

\subsection{Health Monitoring}

Fitness armbands and smart watches is something that has gained much popularity in the recent years. These devices have quite strict requirements when it comes to size, so there is often not much room for a battery. For such applications it is therefore crucial to use ultra low power sensors and efficient data acquisition techniques. However, the sensors also need to have good enough performance to acquire useful data, such as the number of steps taken, heart rate information and so forth.  

\subsection{Vibration Detection}

For some construction applications it can be very beneficial to be able to monitor the vibrations inside the structure itself. Ultra low power sensors can for instance be submersed inside concrete walls and transmit during the entire life-span of the building. From this application one can really see the problem with changing batteries. As one would literally need to tear down a wall to change them. 

\subsection{Pedometer}

...

\subsection{Navigation}

...