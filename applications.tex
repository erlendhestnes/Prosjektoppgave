\chapter{Motion Based IoT Applications}
\label{chap:applications}

The proposed reference board presented in Chapter \ref{chap:reference} is going to be used to showcase different IoT applications that could possibly benefit from utilizing a low power accelerometer. This chapter presents some possibilities with regards to creating such applications. 

\section{Types of measurement}

Consumer grade MEMS accelerometers can typically measure from a range of $\pm2g$ to a range of $\pm16g$, as seen in the overview from Chapter \ref{chap:overview}. This measurement range is perfectly suited for most everyday use-cases. To put things in perspective, we can look at the acceleration of a few different events (from \cite{g-force}). Light vibrations for instance are in the order of 0.014 - 0.039g's, while the highest recorded earthquake was at its peak 2.99g's. The Bugatti Veyron supercar pulls 1.55g's when doing 0-100km/h in 2.4 seconds, while a F-16 fighter jet is at most able to pull 9g-12g for limited amount of time during a barrel roll. A baseball struck by a baseball bat is on the other hand in the order of 3000g's, which is way above the measurement range for these devices. An acceleration of such magnitude may even in some cases destroy the MEMS element. 

Based on the figures presented in the section above, the most relevant measurement range for an IoT application will be in the range of $\pm2g$ to $\pm4g$. In this range we find both motion and vibration. The former is defined in this thesis as a slow moving event such as the movement of game controller or smart phone, while the latter is defined as oscillations that occur about an equilibrium point. Both motion and vibration can be useful for a vast amount of IoT applications, whereas some of them will be presented in the sections below. 

\subsection{Motion detection}

Motion detection can in its simplest form be used as a trigger to wake an external system, this solution can for example be a viable substitute for a mechanical on/off switch. For such a configuration we may only care about power consumption, since the collected data is not going to be used for any control input. An example here could be to fuse such a sensor, together with an MCU with radio and temperature sensor, inside a coffee mug. The motion sensor could then detect when the mug was being picked up. Upon such an event it could take a temperature sample and transmit the data to your phone and perhaps give you a pre-drink warning if the beverage was to hot. This example might seem a little frivolous, but it shows some very important benefits of using a low power motion activated trigger. The clear benefit of fusing the system inside the mug is that we can safely have it inside the dishwasher without damaging the electronics. The downside of fusing the system inside the mug is that it will be impossible to change the battery without breaking the mug. We could of course use a form of wireless charging and a re-chargeable battery for this system. However, the system would then need to include additional battery management circuitry which would increase both area and cost. For most users it would also be impractical if the mug would need to be recharged before use. For the sake of the example, a non-rechargeable power source is therefore assumed. This means that the power consumption needs to be ultra low, so that the battery can last for the entire lifespan of the mug. The mug will probably spend most of its lifespan inside one of the kitchen cupboards doing nothing. It is first when you pick up the mug that you actually would want it to do something interesting, like measuring the temperature of your desired beverage. This could be achieved by incorporating a mechanical on/off switch somewhere on the mug, but that would be harder to make waterproof and would require more effort from the user. It is easy to see how a motion activated switch would solve both of these problems. 

For an application where control input is a requirement, as well the motion activated trigger functionality, we would be interested in an accelerometer that uses low power in wake-up mode, but are able to sample with precision when it is required. Lets say we wanted to design a self stabilizing coffee mug. This could for instance be very useful for people suffering from Parkinson's disease. A realization of such a product would require more precision from the motion sensor than for the simple temperature example, since the actual data from the motion sensor is going to be used to generate the counter movements to achieve stabilization. Similar to the temperature example above, the self stabilizing coffee mug would only need to enter stabilization mode when it actually was being picked up. 

%\subsection{Health Monitoring}

%Fitness armbands and smart watches is something that has gained much popularity in the recent years. These devices have quite strict requirements when it comes to size, so there is often not much room for a battery. For such applications it is therefore crucial to use ultra low power sensors and efficient data acquisition techniques. However, the sensors also need to have good enough performance to acquire useful data, such as the number of steps taken, heart rate information and so forth.  

%Active implants placed inside the body is also something that there has been a substantial amount of research on. These implants can for instance be used to monitor blood sugar levels, heart rhythms or oxygen levels. 

\subsection{Vibration detection}

Vibration detection can be interesting for a vast amount of IoT applications. For construction work, it could for instance be very beneficial to be able to monitor vibrations inside a structure. This could prove to be especially useful for areas that suffers from earthquakes, as vibration data from the sensor nodes could help engineers better learn how to create structures that can withstand such massive vibrations. 

One possible realization of such a system, could be to submerge wireless sensor nodes inside the concrete walls. The sensor nodes could then communicate with each other and with the Internet through a gateway. If a sufficient amount of nodes were placed inside the walls, they could help to better visualize the mechanical stress inside the building walls.

One of the biggest challenges with such a system would be the power consumption. The sensor nodes would need to have batteries that would last for decades, as one would literally need to tear down a wall to change them. Ultra-low power accelerometers would therefore be well suited for such an application. 

\subsection{Inertial navigation}

Inertial navigation uses a combination of motion sensors (accelerometers), rotation sensors (gyroscopes) and magnetometers to calculate the current position of a moving object based on a previously determined position. This calculation process is often referred to as dead-reckoning \cite{dead_reckoning}. A typical example is advanced navigation systems for automotive use. Such systems initially uses GPS (Global Positioning System) to determine the cars position. But if the GPS signal is lost, the system is able to switch over to inertial navigation. The speed of the car as well as the movement from the steering wheel is then used to estimate the cars position based on the previous GPS fix. This makes it possible for the navigation system to work inside a tunnel for instance. 

Dead-reckoning is based on path integration and is therefore a subject for cumulative errors, meaning that the system becomes less and less precise over time. Fast integration as well as precision sensors are required to minimize this error. Inertial navigation might not therefore be the best suited application for a low power device. It might however be possible to implement a simpler scheme, where goal is to only estimate the path of moving object for couple of minutes. In combination with some radio reference points (anchors), this could be used to achieve a simple form of indoor navigation.  