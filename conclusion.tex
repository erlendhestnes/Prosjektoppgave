%This is the Preface
%%=========================================
\chapter{Conclusion}

A study of MEMS based accelerometers has been carried out. The work has covered a wide range of topics with regards to the subject. Starting with the fundamental working principles of an accelerometer, continuing with different MEMS based implementations of an accelerometer. This has given useful information about some of the fundamental limitations, as well as some clear benefits with using the technology for future IoT applications. 

%mye å velge i, alle gode på sitt område, ut ifrra krieriene så har jeg valgt en...

This background theory has been the foundation for an analysis of five of the most low power accelerometers that are currently available on the market. The analysis has revealed that selecting a component for an embedded system is highly application specific, which makes it quite difficult to declare a winner that will be the obvious choice in every situation. The analysis has on the other hand showed that the ADXL362 from Analog Devices proves to be the most low power device currently available. This device also proves to be quite versatile, having a lot of different configuration options that can trade power for precision. This makes the accelerometer perfectly suited for a number of motion based IoT applications, whereas three  been proposed in this thesis. 

\section{Further Work}

A custom reference board using the ADXL362 from Analog Devices was designed as a part of this thesis. Several IoT applications has been proposed for this reference board in Chapter \ref{chap:applications}. For a future master thesis, the plan is to implement some of these applications using the reference board. 
