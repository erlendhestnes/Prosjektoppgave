%This is the Preface
%%=========================================
\chapter{Conclusion}

%mye å velge i, alle gode på sitt område, ut ifrra krieriene så har jeg valgt en...
A study of MEMS based accelerometers has been carried out. The work has covered a wide range of topics with regards to the subject. Starting with the fundamental working principles of an accelerometer, continuing with different MEMS based implementations of an accelerometer. This has given useful information about some of the fundamental limitations, as well as some clear benefits with using the technology for future IoT applications. This background theory has been the foundation for an analysis of five of the most low power accelerometers that are currently available on the market. The analysis has revealed that the ADXL362 from Analog Devices proves to be the most low power device currently available. This device also proves to be quite versatile, having many different configuration options that can trade power for precision. A custom reference board using the ADXL362 and a nRF51 SoC has been designed as part of this work. A state of the art data acquisition scheme using low power peripherals found in the nRF51 has been proposed as well. Three different IoT applications has been suggested for the reference board.

\section{Further Work}

A custom reference board using the ADXL362 from Analog Devices was designed as a part of this thesis. Three IoT applications was proposed for this reference board in Chapter \ref{chap:applications}. In a future master thesis, one of these three applications may be explored further by using the reference board. 

The reference board designed in this thesis, along with the three proposed IoT applications was planned to be used for a future master thesis.
