\chapter{Accelerometer Overview}

\section{Comparison of Ultra-Low Power Accelerometers}

The accelerometers presented in Table \ref{tab:accel_comparison} represent the most low power, consumer grade, MEMS accelerometers that are currently available on the market. The specifications are gathered from the data sheets of each individual component. All of the listed accelerometers uses the surface micromachined fabrication process. This is not surprising, as this process has become the de-facto standard for cheap, low-power MEMS designs.  

\begin{figure}[h]
\begin{center}
    \resizebox{\textwidth}{!} {
    \begin{tabular}{ | l | l | l | l | l | l |}
    \hline
    Device & ADXL362 & MMA8491QR1 & MC3610 & LIS3DH & KX123 \\ \hline
    
    Manufacturer & Analog Devices & Freescale Semiconductor & mCube & STMicroelectronics & Kionix \\ \hline
    
    Supply Voltage & 1.6-3.5V  & 1.95-3.6V & 1.6-3.6V & 1.71-3.6V & 1.8-3.6V \\ \hline
    
    Standby current & 10$\si{\nano\ampere}$ ,$V_s = 2.0 V$ & 1.8$\si{\nano\ampere}$ ,$V_s = 2.8 V$ & 400$\si{\nano\ampere}$ ,$V_s = 1.8 V$ & 500$\si{\nano\ampere}$ ,$V_s = 2.5 V$ & 900$\si{\nano\ampere}$ ,$V_s = 2.5 V$ \\ \hline
    
    Max ODR & 400Hz & 800Hz & 400Hz & 1.25/5kHz \footnote[3] & 25.6kHz \\ \hline
    
    ODR current & 1.8$\si{\micro\ampere}$ / 13$\si{\micro\ampere}$ \footnote[2] & 40$\si{\micro\ampere}$ \footnote[1] & 4.7$\si{\micro\ampere}$ / 14$\si{\micro\ampere}$, (FIFO On) \footnote[4] & 20$\si{\micro\ampere}$ / 10$\si{\micro\ampere}$ \footnote[3] & 21$\si{\micro\ampere}$ \\
    
    & $V_s = 2.0 V$, 100Hz ODR & $V_s = 2.8 V$, 100Hz ODR & 3.2$\si{\micro\ampere}$ / 8.4$\si{\micro\ampere}$, (FIFO Off) \footnote[4] & $V_s = 2.5 V$, 100Hz ODR  & $V_s = 2.5 V$, 100Hz ODR \\
    
    & & & $V_s = 1.8 V$, 50Hz ODR & &  \\ \hline
    
    Sensitivity & 1mg/LSB (@ $\pm$2g) & 1mg/LSB (@ $\pm$1-8g) & - & 1mg/LSB (@ $\pm$2g) & 16mg/LSB (@ $\pm$2g)\\ \hline

    Spectral Noise (X,Y) & 550$\si{\micro}$g/$\sqrt{Hz}$ / 250$\si{\micro}$g/$\sqrt{Hz}$ \footnote[2] & 1626ug/$\sqrt{Hz}$ \footnote[6] & 280$\si{\micro}$g/$\sqrt{Hz}$ & 220ug/$\sqrt{Hz}$ / N.A. \footnote[3] & \\ 
    
    & $V_s = 2.0 V$,100Hz ODR & $V_s = 2.8 V$,100Hz ODR & $V_s = 1.8 V$,50Hz ODR & $V_s = 2.5 V$,100Hz ODR & $V_s = 2.5 V$,50Hz ODR \\ \hline
    
    Digital Resolution & 12-bit & 14-bit & 14-bit & 16-bit & 16-bit \\ \hline
    
    Interface & SPI & I2C & SPI/I2C & SPI/I2C & SPI/I2C \\ \hline
    
    Measurement range & $\pm$2,4,8g & $\pm$1-8g (full scale) & $\pm$2,4,8,12,16g & $\pm$2,4,8,16g & $\pm$2,4,8g \\ \hline
    
    Additional features & FIFO (512 Samples) & Ultra-fast response time & FIFO (32 Samples) & FIFO (32 Samples) & FIFO (1024 Samples) \\
    
    & 270nA Motion detect mode  & 3x Interrupt pins  & 600nA Motion detect mode & Motion detect, free fall & Motion and tap detect   \\
    
    & 2x Interrupt pins  & Automatic power-saving & 1x Interrupt pin & 2x Interrupt pins & 2x Interrupt pins \\
    
    & Temperature Sensor  &  &  & Temperature Sensor &  \\ \hline
    
    \end{tabular}
    }
    \caption{Comparison of ultra-low power MEMS accelerometers currently on the market. All figures represent typical values, unless stated otherwise.}
    \label{tab:accel_comparison}
\end{center}
\end{figure}

\footnotetext[1]{Specified at 400nA/Hz in datasheet. 400nA * 100 = 40$\si{\micro\ampere}$}
\footnotetext[2]{Normal mode/Ultralow noise mode}
\footnotetext[3]{Normal mode/Low power mode}
\footnotetext[4]{Low power mode/Precision mode}
\footnotetext[5]{Low power mode/High resolution mode}
\footnotetext[6]{$PSD = Nrms / BW$}

\subsection{ADXL362}

The ADXL362 from Analog Devices is currently the lowest power accelerometer in the industry, according to \cite{analog12}. At a supply voltage of $V_s = 2.0 V$ it uses only 10nA in standby mode and 1.8$\si{\micro\ampere}$ at a ODR of 100Hz. It also has a lot of power-saving features, such as 270nA wake-up mode, two interrupt pins and a deep embedded FIFO that can hold 512 samples. The FIFO enables the accelerometer to autonomously collect samples for an extended period of time, without involving the host controller. 
The ADXL362 has a SPI interface which can handle a bus clock frequency of 8 MHz. This makes it possible for the host controller to acquire data at a very short amount of time, which is also an important aspect in a low-power design.

However, as seen from Table \ref{tab:accel_comparison} this ultra low power consumption comes at cost. Of all of the compared accelerometers, the ADXL362 has the worst spectral noise density, the lowest resolution and the smallest measurement range. That being said, the ADXL362 also features a ultralow noise mode. This mode uses almost ten times more current, but halves the spectral noise, making the accelerometer on pair with it's competitors. Even more interestingly is that even with this ultralow noise mode enabled, the ADXL362 still has the lowest current draw for a 100Hz ODR. The ODR current was only specified at 50Hz ODR for the MC3610, LIS3DH and KX123, but one can assume that current would be around twice that of the 50Hz ODR.

\subsection{MC3610}
mCube is a relatively new and small MEMS company in this comparison. Yet, they have managed to develop av very good and low-power solution with the MC3610. The accelerometer has the largest measurment range, decent resolution and many additional features such as an embedded FIFO and a motion activated wake-up mode. It is in the top league when it comes to current consumption, with only a current draw of 3.2$\si{\micro\ampere}$ in low-power mode. It does however use relatively much current when in standby.

\subsection{LIS3DH}

The LIS3DH from STMicroelectronics is without doubt the most feature rich accelerometer in this comparison. The accelerometer features motion and free fall detection, two programmable interrupt pins, an integrated temperature sensor and an embedded FIFO. The FIFO is able to hold 32 samples, but only at resolution of 10-bits. The LIS3DH also has something STMicroelectronics calls 4D/6D orientation detection, which essentially enables the accelerometer to generate an interrupt when the device is stable in a know direction. The LIS3DH also has the highest digital resolution of 16-bit and the best spectral noise density. The current consumption in sampling mode is competitive with the ADXL362. However, ST does not provide any numbers on the noise spectral density for the low power mode. They only state that the noise will increase when this mode is entered. The standby current is also relatively high when compared to the ADXL362 and the MMA8491QR1. 

\subsection{KX123}

The KX123 from Kionix combines a very high resolution together with really large FIFO buffer of 2048 bytes. It also has some nice features such as tap and double tap detection, as well as motion activated wake-up modes.

\subsection{MMA8491QR1}

The MMA8491QR1 from Freescale Semiconductor. It features interrupt pins for each acceleration axis, which is being used for tilt detection. It also has ultra fast data output time of only 700$\si{\micro\second}$. It has an ultra low standby current of only 1.8nA, which is by far the best of all the compared accelerometers in Table \ref{tab:accel_comparison}. It is also has a full scale measurement range from $\pm$1-8g's.