\chapter{Accelerometer Overview}
\label{chap:overview}

This section presents an overview of five of the most low power, consumer grade, MEMS accelerometers that are currently available on the market.

\section{Comparison of Ultra-Low Power Accelerometers}

Five accelerometers from five different semiconductor vendors are presented in Table \ref{tab:accel_comparison}. The specifications are gathered from the datasheets of each individual component. All of the listed accelerometers uses the surface micromachined fabrication process. This is not surprising, as this process has become the de-facto standard for cheap, low-power MEMS designs.  

\begin{figure}[h]
\begin{center}
    \resizebox{\textwidth}{!} {
    \begin{tabular}{ | l | l | l | l | l | l |}
    \hline
    \textbf{Device} & \textbf{ADXL362} & \textbf{MMA8491QR1} & \textbf{MC3610} & \textbf{LIS3DH} & \textbf{KX123} \\ \hline
    
    \textbf{Manufacturer} & Analog Devices & Freescale Semiconductor & mCube & STMicroelectronics & Kionix \\ \hline
    
    \textbf{Supply Voltage} & 1.6-3.5V  & 1.95-3.6V & 1.6-3.6V & 1.71-3.6V & 1.8-3.6V \\ \hline
    
    \textbf{Shutdown current} & 10$\si{\nano\ampere}$ ,$V_s = 2.0 V$ & 1.8$\si{\nano\ampere}$ ,$V_s = 2.8 V$ & 400$\si{\nano\ampere}$ ,$V_s = 1.8 V$ & 500$\si{\nano\ampere}$ ,$V_s = 2.5 V$ & 900$\si{\nano\ampere}$ ,$V_s = 2.5 V$ \\ \hline
    
    \textbf{Max ODR} & 400Hz & 800Hz & 400Hz & 1.25/5kHz \footnote[3] & 25.6kHz \\ \hline
    
    \textbf{ODR current} & 1.8$\si{\micro\ampere}$ / 13$\si{\micro\ampere}$ \footnote[2] & 20$\si{\micro\ampere}$ \footnote[1] & 4.7$\si{\micro\ampere}$ / 14$\si{\micro\ampere}$, (FIFO On) \footnote[4] & 20$\si{\micro\ampere}$ / 10$\si{\micro\ampere}$ \footnote[3] & 21$\si{\micro\ampere}$ \\
    
    & $V_s = 2.0 V$, 100Hz ODR & $V_s = 2.8 V$, 100Hz ODR & 3.2$\si{\micro\ampere}$ / 8.4$\si{\micro\ampere}$, (FIFO Off) \footnote[4] & $V_s = 2.5 V$, 100Hz ODR  & $V_s = 2.5 V$, 100Hz ODR \\
    
    & & & $V_s = 1.8 V$, 50Hz ODR & &  \\ \hline
    
    \textbf{Sensitivity} & 1mg/LSB (@ $\pm$2g) & 1mg/LSB (@ $\pm$1-8g) & - & 1mg/LSB (@ $\pm$2g) & 16mg/LSB (@ $\pm$2g)\\ \hline

    \textbf{Spectral Noise (X,Y)} & 550$\si{\micro}$g/$\sqrt{Hz}$ / 250$\si{\micro}$g/$\sqrt{Hz}$ \footnote[2] & 1626ug/$\sqrt{Hz}$ \footnote[6] & 280$\si{\micro}$g/$\sqrt{Hz}$ & 220ug/$\sqrt{Hz}$ / N.A. \footnote[3] & \\ 
    
    & $V_s = 2.0 V$,100Hz ODR & $V_s = 2.8 V$,100Hz ODR & $V_s = 1.8 V$,50Hz ODR & $V_s = 2.5 V$,100Hz ODR & $V_s = 2.5 V$,50Hz ODR \\ \hline
    
    \textbf{Digital Resolution} & 12-bit & 14-bit & 14-bit & 16-bit & 16-bit \\ \hline
    
    \textbf{Interface} & SPI & I2C & SPI/I2C & SPI/I2C & SPI/I2C \\ \hline
    
    \textbf{Measurement range} & $\pm$2,4,8g & $\pm$1-8g (full scale) & $\pm$2,4,8,12,16g & $\pm$2,4,8,16g & $\pm$2,4,8g \\ \hline
    
    \textbf{Additional features} & FIFO (512 Samples) & Ultra-fast response time & FIFO (32 Samples) & FIFO (32 Samples) & FIFO (1024 Samples) \\
    
    & 270nA Motion detect mode  & 3x Interrupt pins  & 600nA Motion detect mode & Motion detect, free fall & Motion and tap detect   \\
    
    & 2x Interrupt pins  & Automatic power-saving & 1x Interrupt pin & 2x Interrupt pins & 2x Interrupt pins \\
    
    & Temperature Sensor  &  &  & Temperature Sensor &  \\ \hline
    
    \end{tabular}
    }
    \caption{Comparison of ultra-low power MEMS accelerometers currently on the market. All figures represent typical values, unless stated otherwise.}
    \label{tab:accel_comparison}
\end{center}
\end{figure}

\footnotetext[1]{Specified at 400nA/Hz in datasheet. 400nA * 50 = 20$\si{\micro\ampere}$}
\footnotetext[2]{Normal mode/Ultralow noise mode}
\footnotetext[3]{Normal mode/Low power mode}
\footnotetext[4]{Low power mode/Precision mode}
\footnotetext[5]{Low power mode/High resolution mode}
\footnotetext[6]{$PSD = Nrms / BW$}

\subsection{ADXL362}

The ADXL362 from Analog Devices is currently the lowest power accelerometer in the industry, according to \cite{analog12}. At a supply voltage of $V_s = 2.0 V$ it uses only 10nA in shutdown mode and 1.8$\si{\micro\ampere}$ at a ODR of 100Hz. It also has a lot of features, such as 270nA wake-up mode, two interrupt pins, temperature sensor and a deep embedded FIFO that can hold 512 12-bit samples. The FIFO enables the accelerometer to autonomously collect samples for an extended period of time, without involving the host controller.

The FIFO in the ADXL362 can be configured to either hold 170 sample sets of concurrent 3-axis data or 128 sample sets of concurrent 3-axis and temperature data. The FIFO also has lot of configuration options,  

The ADXL362 has a SPI interface which can handle a bus clock frequency of 8 MHz. This makes it possible for the host controller to acquire data at a very short amount of time, which is also an important aspect in a low-power design.

However, as seen from Table \ref{tab:accel_comparison} this ultra low power consumption comes at cost. Of all of the compared accelerometers, the ADXL362 has the worst spectral noise density, the lowest resolution and the smallest measurement range. That being said, the ADXL362 also features a ultralow noise mode. This mode uses almost ten times more current, but halves the spectral noise, making the accelerometer on pair with it's competitors. Even more interestingly is that even with this ultralow noise mode enabled, the ADXL362 still has the lowest current draw for a 100Hz ODR. The ODR current was only specified at 50Hz ODR for the MC3610, but one can assume that current would be around twice that of the 50Hz ODR, as the current consumption is usually proportional to the ODR.

\subsection{MMA8491QR1}

The MMA8491QR1 from Freescale Semiconductor has by far the lowest shutdown current of all the compared accelerometers. It also features interrupt pins for each acceleration axis, which can be used for advanced tilt detection.

Freescale specifies that the current consumption for sampling is 400nA/Hz. They don't however specify whether this number is ODR or bandwidth. By assuming that this number refers to bandwidth, the current consumption of 100Hz ODR is 20$\si{\micro\ampere}$. 

It also has ultra fast data output time of only 700$\si{\micro\second}$.

It is also has a full scale measurement range from $\pm$1-8g's.

Some drawback with this component is that it requires a relatively high supply voltage of 1.95V, and that is only has a I2C interface.

The component has otherwise very few configuration options, as much of the power saving techniques are automated according to Freescale.

\subsection{MC3610}
mCube is a relatively new and small MEMS manufacturer in this comparison. Yet, they have managed to develop av very good and low-power solution with the MC3610. The accelerometer has the most versatile measurement range, decent resolution of 14-bit and many additional features such as an embedded FIFO and a motion activated wake-up mode. It is in the top league when it comes to current consumption. However, there are some restrictions when it comes to using the component at lowest power settings. The FIFO does for instance use some additional current, and ultra low power mode only works at some specific ODR

The shutdown current is also relatively high.

\subsection{LIS3DH}

The LIS3DH from STMicroelectronics is without doubt the most feature rich accelerometer in this comparison. The accelerometer comes with motion and free fall detection, two programmable interrupt pins, an integrated temperature sensor and an embedded FIFO. The FIFO is able to hold 32 samples, but only at resolution of 10-bits. The LIS3DH also has something STMicroelectronics calls 4D/6D orientation detection, which essentially enables the accelerometer to generate an interrupt when the device is stable in a known direction. The LIS3DH also has a high digital resolution of 16-bit and the best spectral noise density. The current consumption in sampling mode is competitive with the ADXL362 if the low power mode is selected. However, STMicroelectronics does not provide any numbers on the noise spectral density for the low power mode. They only state that the noise will increase when this mode is entered. The shutdown current is also relatively high when compared to the ADXL362 and the MMA8491QR1. 

\subsection{KX123}

The KX123 from Kionix combines a very high digital resolution of 16-bits together with the highest ODR of all the accelerometers in the comparison. The device also has a large FIFO buffer of 2048 bytes, which are then able to hold 1024 full resolution samples. The FIFO also has lots of configuration options. 

The device does however have the highest shutdown current. 

The ODR current is 21 $\si{\micro\ampere}$, which is on pair with the MMA8491QR1 and the LIS3DH. It only has three measurement ranges, which is the same as the ADXL362.

The device is quite feature rich, with single- and double tap detection as well as free fall detection.

It also has wake-up mode, but Kionix does not state how much current it uses.