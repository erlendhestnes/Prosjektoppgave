%This is the Preface
%%=========================================
\addcontentsline{toc}{section}{Abstract}
\begin{center}
\section*{Abstract}
\end{center}

Internet of Things (IoT) applications often involve the use of sensors. Motion sensing can be used in a vast amount of applications and are therefore interesting in high volume IoT products. Providing energy to such devices is often challenging. It is therefore essential for such products to have sensors that consumes as little power as possible. 

This specialization project presents an analysis of five commercially available microelectromechanical (MEMS) accelerometers. The analysis focuses primarily on finding the sensor that is best suited for a range of ultra-low power IoT applications. The best sensor is chosen to be a part of a custom reference board, of which was also designed as a part of this work. The reference board is planned to be used for a later Master thesis, where the goal is to further explore different IoT application areas for ultra-low power motion sensors. Some of these possible areas are also presented in this work. 

From the accelerometer analysis in this project, it has been showed that the ADXL362 from Analog Devices currently is the lowest power accelerometer available. The same device is also versatile, with many different configuration options. This makes it suited for a wide range of IoT applications. Some of the most relevant applications are motion detection, vibration detection and inertial navigation.  

The project is carried out as feasibility study for Disruptive Technologies AS in conjunction with the Norwegian University of Science and Technology (NTNU).