%This is the Preface
%%=========================================
\addcontentsline{toc}{section}{Abstract}
\section*{Abstract}

Internet of Things (IoT) applications very often involve the use of sensors. Motion sensing can be used in a vast amount of applications and are therefore very interesting in high volume IoT products. Replacing a battery in such an application is often undesirable, and may even in some cases be close to impossible. It is therefore essential for such applications to have sensors that consumes as little power as possible. This project thesis explores different techniques to acquire data from a MEMS accelerometer at the lowest possible power consumption.

A custom made development board, fitted with an ultra-low power MEMS accelerometer and a Bluetooth System-On-Chip (SoC), was used to test some of the different power saving techniques that is available today. 

The project is carried out for a Norwegian company called Disruptive Technologies in conjunction with the Norwegian University of Science and Technology.

\begin{center}
Trondheim, 2015-12-16\\[1pc]
\begin{figure}[h]
\centering
\includegraphics[scale=0.5]{fig/underskrift.png}
\label{fig:underskrift}
\end{figure}
Erlend Hestnes
\end{center}